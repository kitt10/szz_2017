\chapter{Aplikovaná kybernetika [AKSZ]}

\section{Umělá inteligence [UI]}

\begin{table}[H]
\centering
\begin{tabular}{p{4cm} p{12cm}}
\textit{vyučující:}             & Prof. Ing. Josef Psutka, CSc. \\
								 & Ing. Aleš Pražák, Ph.D. \\
\textit{ročník/semestr studia:} & 2.ročník/ZS \\
\textit{datum zkoušky:}         & X. X. 2012 \\
\textit{hodnocení:}             & 1 \\
\textit{cíl předmětu (STAG):}   & \\
\multicolumn{2}{p{16cm}}{Cílem předmětu je seznámit studenty se základními problémovými oblastmi umělé inteligence (UI) a naučit je aplikovat vybrané metody řešení úloh, reprezentace znalostí v UI a hraní her.}
\end{tabular}
\end{table}

\subsection{Metody řešení úloh v UI}

\subsection{Logické formalizmy pro reprezentaci znalostí. Predikátový počet 1. řádu. Rezoluční metoda.}

\subsection{Produkční systém. Báze znalostí a báze dat. Dopředné a zpětné šíření.}

\subsection{Síťové formalizmy pro reprezentaci znalostí. Sémantické sítě. Rámce. Scénáře.}

\subsection{Metody hraní her v UI. Procedura minimax, alfa-beta prořezávání.}

\section{Modelování a simulace 1 [MS1]}

\begin{table}[H]
\centering
\begin{tabular}{p{4cm} p{12cm}}
\textit{vyučující:}             & Ing. Václav Hajšman, Ph.D. \\
								 & Ing. Jindřich Liška, Ph.D. \\
								 & Ing. Miloš Fetter \\
\textit{ročník/semestr studia:} & 2.ročník/ZS \\
\textit{datum zkoušky:}         & X. X. 2012 \\
\textit{hodnocení:}             & 1 \\
\textit{cíl předmětu (STAG):}   & \\
\multicolumn{2}{p{16cm}}{Cílem předmětu je seznámit studenty se základními principy modelování dynamických systémů.}
\end{tabular}
\end{table}

\subsection{Systém, model, modelování, simulace, systémová analýza.}

\subsection{Modelování systému diskrétních událostí, diskrétní simulace.}

\subsection{Simulační experiment, studie, analýza rizika, náhoda v simulačních úlohách.}

\subsection{Modelování v netechnických oborech (kompartmenty, buněčné automaty, ...).}

\subsection{Konstrukce modelů na základě měření, zpracování signálu v časové, frekvenční a časo-frekvenční oblasti, modely periodických procesů.}

\subsection{Modely vibrací a kmitání, experimentální modální analýza.}

\subsection{Generování náhodných čísel, metoda Monte Carlo a odhad přesnosti simulačních výsledků.}

\section{Programové prostředky řízení [PP]}

\begin{table}[H]
\centering
\begin{tabular}{p{4cm} p{12cm}}
\textit{vyučující:}             & Ing. Pavel Balda, Ph.D. \\
\textit{ročník/semestr studia:} & 3.ročník/LS \\
\textit{datum zkoušky:}         & X. X. 2014 \\
\textit{hodnocení:}             & 1 \\
\textit{cíl předmětu (STAG):}   & \\
\multicolumn{2}{p{16cm}}{Cílem předmětu je naučit studenty aplikovat některé vybrané techniky programování řídicích a informačních systémů především prostředky jazyka C\#. V rámci předmětu je podána klasifikace operačních systémů a jejich základní vlastnosti. Dále je vysvětlena hierarchie programového vybavení typických řídicích systémů od čidel a akčních členů až po podnikové systémy.}
\end{tabular}
\end{table}

\subsection{Architektura podnikových řídicích systémů; používané programovací jazyky.}

\subsection{Architektura .NET Frameworku; řízený modul, metadata, běh řízeného kódu.}

\subsection{Jazyk C Sharp: hodnotové a referenční typy; jednoduché typy, implicitní konverze; výrazy a operátory; příkazy; výjimky.}

\subsection{Jazyk C Sharp: Členy a přístup k nim; jmenné prostory; třídy, metody, vlastnosti, konstruktory, destruktory; struktury; pole; delegáty; atributy.}

\subsection{Softwarové komponenty: DLL, RPC, COM; interface; OPC.}

\subsection{Operační systémy: procesy a thready, synchronizace, deadlock, inverze priorit; správa paměti; vstupně-výstupní systém, programované vstupy/výstupy, přerušení, DMA, ovladače zařízení; souborové systémy.}

\subsection{Operační systémy reálného času: statické a dynamické plánovací algoritmy.}

\subsection{Struktury vzdálených a virtuálních laboratoří.}

\section{Převodníky fyzikálních veličin [PFV]}

\begin{table}[H]
\centering
\begin{tabular}{p{4cm} p{12cm}}
\textit{vyučující:}             & Ing. Liber Jelínek Ph.D. \\
\textit{ročník/semestr studia:} & 4.ročník/LS \\
\textit{datum zkoušky:}         & 16. 6. 2016 \\
\textit{hodnocení:}             & 2 \\
\textit{cíl předmětu (STAG):}   & \\
\multicolumn{2}{p{16cm}}{Cílem předmětu je seznámit studenty se základními principy, vlastnostmi a modely senzorů a akčních členů pro potřeby automatizace, monitorování a diagnostiky.}
\end{tabular}
\end{table}

\subsection{Struktura a parametry senzorů pro automatizaci, statické a dynamické modely a chyby, metody snižování chyb senzorů.}

\subsection{A/D a D/A převodníky, obvody pro úpravu signálů, frekvenční filtry.}

\subsection{Senzory teploty a tepla, obvody pro měření odporu, kapacity, indukčnosti a frekvence.}

\subsection{Senzory polohy a vzdálenosti (odporové, indukční, kapacitní, ultrazvukové, optické).}

\subsection{Senzory síly, hmotnosti, deformace, tlaku, rychlosti, zrychlení a vibrací (tenzometrické, piezoelektrické, kapacitní a elektrodynamické).}

\subsection{Senzory průtoku, množství, hustoty, viskozity, koncentrace a chemického složení.}

\subsection{Elektrické akční členy a jejich budiče (stejnosměrné, střídavé, krokové motory, PWM zesilovače, frekvenční měniče).}

\subsection{Hydraulické a pneumatické akční členy (pracovní a řídicí mechanizmy a zdroje tlakového média).}