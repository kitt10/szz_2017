\chapter{Teorie řízení [TŘSZ]}

\section{Lineární systémy 1-2 [LS1], [LS2]}

\begin{table}[H]
\centering
\begin{tabular}{p{4cm} p{12cm}}
\textit{vyučující:}             & Doc. Ing. Jiří Melichar, CSc. \\
								 & Ing. Martin Čech, Ph.D. \\
								 & Ing. Jiří Mertl, Ph.D. \\
\textit{ročník/semestr studia:} & 2.ročník/ZS-LS \\
\textit{datum zkoušky:}         & X. 1. 2013/X. X. 2013 \\
\textit{hodnocení:}             & 1/2 \\
\textit{cíl předmětu (STAG):}   & \\
\multicolumn{2}{p{16cm}}{LS1: Student by měl získat přehled o typech, struktuře a chování reálných dynamických systémů, obeznámit se s metodikou tvorby matematických modelů reálných dynamických systémů a s metodami analýzy jejich vlastností a chování v časové i frekvenční oblasti. Student by měl také porozumět základním principům řízení dynamických systémů a metodám pro získávání potřebných dat z reálných procesů. 

Cílem předmětu LS2 je, aby student:
\begin{itemize}
\item získal přehled o klasických regulačních úlohách, o struktuře regulačních obvodů a o základních typech dynamických i nedynamických regulátorů;
\item dokázal analyzovat reálnou regulační úlohu v její celistvosti, uměl formulovat požadavky na kvalitu regulace v časové i frekvenční oblasti při současném respektování všech omezení;
\item byl schopen použít vhodné metody pro návrh spojitých i číslicových regulátorů a získávat potřebná data z reálného procesu;
\item byl schopen analýzy nelineárních dynamických systémů a základní orientace v problémech jejich řízení.
\end{itemize}}
\end{tabular}
\end{table}

\subsection{Matematické modely spojitých a diskrétních lineárních dynamických systémů.}
Za model považujeme \textit{stavovou reprezentaci} dynamických systémů a nebo dynamiku vyjádřenou \textit{diferenciálními rovnicemi}. K modelu dojdeme buď \textit{experimentálně} a nebo (raději) na základě \textit{matematicko-fyzikálního} modelování (znalost fyzikálních zákonů z dané vědní oblasti)
\begin{itemize}
\item \textit{klasická mechanika}
\begin{enumerate}[label=(\alph*)]
\item Newtonovská mechanika - silové zákony, rovnováha sil
\item Lagrangeova mechanika - znalost kinetické a potenciální energie + elektromechanická analogie
\end{enumerate}
\item \textit{elektrické obvody} - Kirchhoffovy zákony
\item \textit{hydrodynamika} - rovnice kontinuity, Bernoulliho zákon
\item jiné získání struktury a parametrů modelu
\end{itemize}

Z matematicko-fyzikálního modelování získáme diferenciální rovnice vyššího řádu, které lze převést na soustavu diferenciálních rovnic prvního řádu (lineární či nelineární). Odtud potom získáme stavovou reprezentaci.

Spojitý systém:
\begin{align}
\begin{split}
\dot{x}(t) &= A \cdot x(t) + B \cdot u(t) \\
y(t) &= C^T \cdot x(t) + D \cdot u(t)
\end{split}
\end{align}

Diskrétní systém:
\begin{align}
\begin{split}
x(k+1) &= A \cdot x(k) + B \cdot u(k) \\
y(k+1) &= C^T \cdot x(k) + D \cdot u(k)
\end{split}
\end{align}

\subsubsection*{Newtonova mechanika}
Newtonův zákon síly:
\begin{equation}
\boxed{F = m \cdot a}
\end{equation}
K sestavení stavového modelu se zavedou složky vektoru stavu $ x_1(t), x_2(t) $:
\begin{align}
\begin{split}
y(t) &\coloneqq x_1(t) \\
\dot{y}(t) &= \dot{x}_1(t) \coloneqq x_2(t) \\
\end{split}
\end{align}
Víme, že $ x_1(t) $ odpovídá poloze. Derivace polohy, tedy $ x_2(t) $, odpovídá rychlosti. Zrychlení poté odpovídá derivace rychlosti, tedy $ \dot{x}_2(t) = a = \frac{F(t)}{m} $. Nyní už můžeme sestavit stavový model založený na zákonu síly:
\begin{align}
\begin{split}
\dot{x}(t) &= A \cdot x(t) + b \cdot F(t) = \begin{bmatrix}
0 & 1 \\ 0 & 0
\end{bmatrix} \cdot x(t) + \begin{bmatrix}
0 \\ \frac{1}{m}
\end{bmatrix} \cdot F(t) \\
y(t) &= C^T \cdot x(t) = \begin{bmatrix}
1 & 0
\end{bmatrix} \cdot x(t)
\end{split}
\end{align}

\subsubsection*{Lagrangeova mechanika}
Založena na znalosti kinetické a potenciální energie.

Lagrangeova funkce:
\begin{equation}
L(y, \dot{y}) \coloneqq T(\dot{y}) - V(y)
\end{equation}

Lagrangeova rovnice I. druhu:
\begin{equation}
F(t) = \frac{d}{dt} \frac{\partial L(y, \dot{y})}{\partial \dot{y}} - \frac{\partial L(y, \dot{y})}{\partial y} = \frac{d}{dt} \frac{\partial T(\dot{y})}{\partial \dot{y}} + \frac{\partial V(y)}{\partial y}
\end{equation}

Analogie pro hmotný bod z Newtonova zákonu síly:
\begin{align}
\begin{split}
F_1(t) &= m \cdot \ddot{y}(t) = \frac{d}{dt}(m \cdot \dot{y}) = \frac{d}{dt} \left[ \frac{\partial}{\partial \dot{y}} (\frac{1}{2} \cdot m \cdot \dot{y}^2 ) \right] =  \frac{d}{dt} \frac{\partial T(\dot{y})}{\partial \dot{y}} \\
F_2(t) &= m \cdot g = \frac{\partial}{\partial y} (m \cdot g \cdot y)
\end{split}
\end{align}

\subsubsection*{Další příklady}
\begin{itemize}
\item \textit{Př.: Matematické kyvadlo}
\begin{enumerate}[label=(\alph*)]
\item Newton:
\begin{align*}
\begin{split}
m \cdot a(t) &= m \cdot g \cdot sin(\varphi(t)); \qquad ds(t) = l \cdot d\varphi(t) \\
v(t) &= \frac{ds(t)}{dt} = l \cdot \dot{\varphi}(t) \\
a(t) &= \frac{d}{dt} \cdot v(t) = l \cdot \ddot{\varphi}(t) \\
\to m \cdot l \cdot \ddot{\varphi}(t) &= m \cdot g \cdot sin(\varphi(t)), \, \varphi(0) \\
\ddot{\varphi}(t) &= \frac{g}{l} \cdot sin(\varphi(t))
\end{split}
\end{align*}
\item Lagrange:
\begin{align*}
\begin{split}
T(\cdot) &= \frac{1}{2} \cdot m \cdot v^2 = \frac{1}{2} \cdot m \cdot (V_x^2 + V_y^2) \\
x(t) &= l \cdot sin(\varphi(t)); \qquad y(t) = l - l \cdot cos(\varphi(t)) \\
\dot{x}(t) &\coloneqq V_x(t) = l \cdot cos(\varphi(t)) \cdot \dot{\varphi}(t); \qquad
\dot{y}(t) \coloneqq V_y(t) = l \cdot sin(\varphi(t)) \cdot \dot{\varphi}(t) \\
\to T(\cdot) &= \frac{1}{2} \cdot m \cdot \left[l^2 \cdot cos^2(\varphi(t)) \cdot \dot{\varphi}^2(t) + l^2 \cdot sin^2(\varphi(t)) \cdot \dot{\varphi}^2(t)\right] = \frac{1}{2} \cdot m \cdot l^2 \cdot \dot{\varphi}^2(t)
\end{split}
\end{align*}
\begin{align*}
\begin{split}
V(\cdot) &= m \cdot g \cdot y(t) = m \cdot g \cdot l - m \cdot g \cdot l \cdot cos(\varphi(t)) \\
\to L(\cdot) &= T(\cdot) - V(\cdot) = \frac{1}{2} \cdot m \cdot l^2 \dot{\varphi}^2(t) - m \cdot g \cdot l + m \cdot g \cdot l \cdot cos(\varphi(t))
\end{split}
\end{align*}
Dosadíme do Lagrangeovy rovnice:
\begin{align*}
\begin{split}
\frac{d}{dt} \cdot \frac{\partial L}{\partial \dot{\varphi}(t)} - \frac{\partial L}{\partial \varphi} &= 0 \\
l \cdot \ddot{\varphi}(t) + g \cdot sin(\varphi(t)) &= 0 \\
\to \ddot{\varphi}(t) + \frac{g}{l} \cdot sin(\varphi(t)) &= 0, \, \varphi(0)
\end{split}
\end{align*}
\end{enumerate}
\item \textit{Př.: RLC obvod}

Kirchhoff: $ \boxed{\sum u = 0} $
\begin{align*}
u(t) &= L \frac{di(t)}{dt} + \frac{1}{C} \int i(t) dt = Ri(t) \\
\frac{du(t)}{dt} &= L \frac{d^2i(t)}{dt^2} + \frac{1}{C} i(t) + R \frac{di(t)}{dt}
\end{align*}
Volba stavových proměnných:
\begin{equation*}
x_1(t) \coloneqq i(t); \qquad x_2(t) \coloneqq u_c = \frac{1}{C} \int i(t) dt
\end{equation*}
\begin{equation*}
\to \dot{x}_1(t) = -\frac{R}{L} x_1(t) - \frac{1}{L} x_2(t) + \frac{1}{L} u(t); \qquad \dot{x}_2(t) = \frac{1}{C} x_1(t); \qquad y(t) = R x_1(t)
\end{equation*}
\item \textit{Př.: Interaktivní model dravec-kořist}
\vspace{2.5cm}
\item \textit{Př.: Levitace kuličky v magnetickém poli}
\vspace{2.5cm}
\item \textit{Př.: Stejnosměrný motor řízený do kotvy}
\vspace{3.5cm}
\item \textit{Př.: Jednoduchý tlumič}
\vspace{2.5cm}
\end{itemize}

\subsection{Linearizace nelineárních dynamických systémů, rovnovážné stavy. Harmonická linearizace.}
Pro lineární dynamické systémy (LDS) máme:
\begin{itemize}
\item \textit{rovnovážný stav} $ x_r $ při $ u(t) = 0 \, \forall t \, \iff \, 0 = A \cdot x_r \begin{cases} x_r = 0 & h[A] = n \\ \mathrm{integrator} & h[A] = k < n \end{cases} $
\item \textit{ustálený stav} $ x_r $ při $ u(t) = \mathrm{konst} \, \forall t \, \iff \, 0 = A \cdot x_r + B \cdot u_{konst} $
\end{itemize}
Pro nelineární dynamické systémy (NDS):
\begin{itemize}
\item \textit{rovnovážný stav} $ x_r $ při $ u(t) = 0 \, \forall t \, \iff 0 = f(x_r) \to y_r = g(x_r) $ (obecně více rovnovážných stavů)
\item \textit{ustálený stav} $ x_r $ při $ u(t) = \mathrm{konst} \, \forall t \, \iff \, 0 = f(x_r, u_{konst}) \to y_r = g(x_r) $
\end{itemize}

\subsubsection*{Linearizace NDS}
Bodová (vs. exaktní) linearizace (v okolí rovnovážných či ustálených stavů).

NDS:
\begin{equation*}
\dot{x}(t) = f(x,u); \qquad y(t) = h(x,u)
\end{equation*}
Rovnovážné stavy: $ x_r, u_{konst}, y_r $. Linearizovaný model v okolí $ x_r $ (bude akceptovatelný v "blízkém okolí" rovnovážných stavů):
\begin{align*}
\begin{split}
x(t) &= x_r + \Delta x(t); \qquad y(t) = y_r + \Delta y(t); \qquad u(t) = u_{konst} + \Delta u(t) \\
\dot{x}_r + \Delta \dot{x}(t) &= f(x_r + \Delta x, u_{konst} + \Delta u) = f(x_r, u_{konst} + \frac{\partial f(x,u}{\partial x} \Delta x(t) + \frac{\partial f(x,u)}{\partial u} \\ \Delta y &= \frac{\partial f C}{\partial x} \Delta x + \frac{\partial f C}{\partial u}
\end{split}
\end{align*}

O dynamice LDS či linearizovaného DS rozhodují vlastní čísla matice $ A $: $ \lambda_i(A), \, i=1,...,n $, resp. rozhodují o chování trajektorií systému vzhledem k rovnovážným stavům.
\begin{equation}
\{\lambda_i(A)\}_{i=1}^n = \begin{cases} \mathrm{realna} & \begin{cases} \mathrm{kladna} \\ \mathrm{zaporna} \end{cases} \\ \mathrm{ryze \, imaginarni}  \\ \mathrm{komplexne \, sdruzena} & \begin{cases} Re\{\lambda_i\} > 0 \\ Re\{\lambda_i\} < 0 \end{cases}\\ \lambda_i \neq 0 \to \infty \, \mathrm{rovnovaznych \, stavu} \end{cases}
\end{equation}

\subsubsection*{Typy rovnovážných stavů (LDS II. řádu)}
\begin{enumerate}[label=(\alph*)]
\item \textit{"střed"}: $ \lambda_{1,2} = \pm i $
\vspace{2cm}
\item \textit{"ohnisko"}: $ \lambda_{1,2} = a \pm bi $
\vspace{2cm}
\item \textit{"uzel"}: $ \lambda_{1,2} \in R $
\vspace{2cm}
\item \textit{"sedlo"}: $ \lambda_{1,2} \in R, \, \lambda_1 < 0 \, \land \, \lambda_2 > 0  $
\vspace{2cm}
\end{enumerate}
\subsubsection*{Explicitní řešení stavové rovnice}
\begin{equation}
x(t) = e^{A(t-t_0)} \cdot x(t_0) + \displaystyle{\int_{t_0}^{t}} e^{A(t-\tau)} \cdot \beta \cdot u(\tau) d\tau; \qquad \tau \in \left[t_0, t\right]
\end{equation}
Výpočet stavové matice přechodu $ e^{A(t)} $:
\begin{enumerate}
\item Rozvoj v řádu: $ e^{A(t)} = I + \frac{A \cdot t}{1!} + \frac{A^2 \cdot t^2}{2!} + \cdots = \displaystyle{\sum_{i=0}^{\infty}} A^i \cdot \frac{t^i}{i!} $
\item Využití Laplaceovy transformace: $ e^{At} = \mathscr{L}^{-1}\{(pI-A)^{-1}\} $
\item Každá čtvercová matice může být převedena na diagonální: ($ A = V \cdot D \cdot V^{-1}$)
\begin{equation}
\qquad e^{At} = V \cdot e^{Dt} \cdot V^{-1}
\end{equation}
\item Využití Clay-Hamiltonovy věty (\textit{"Každá čtvercová matice (A) vyhovuje své charakteristické rovnici"}): $ \lambda_i $ jsou řešením charakteristické rovnice pro matici $ A $.
\vspace{3cm}
\end{enumerate}
\subsubsection*{Harmonická linearizace}
Pro analýzu uzavřeného obvodu vycházíme z předpokladů:
\begin{enumerate}
\item V obvodu vznikly ustálené kmity (autooscilace) ze základní frekvencí $ \omega_0 $
\item Lineární systém má charakter dolnofrekvenční propust, $ |F_s(j\omega_0)| >> |F_s(jk\omega_0)|, \, k \geq 2 $.
\item Nelinearita je symetrická vůči nulovému bodu.
\end{enumerate}
Za těchto předpokladů je na výstupu nelinearity \textit{periodický signál} se základní frekvencí $ \omega_0 $ a můžeme jej rozložit ve Fourierovu řadu. Protože chceme nahradit statickou nelinearitu v regulační smyčce tzv. \textit{ekvivalentním přenosem}, budeme tento přenos definovat analogicky jako u frekvenčních přenosů lineárních dynamických systémů. Ekvivalentní přenos statické nelinearity definujeme poměrem výstupního a vstupního signálu nelinearity, respektive poměrem jejich Fourierových obrazů.

\subsection{Vlastnosti lineárních dynamických systémů. Řiditelnost, pozorovatelnost, kriteria. Vnitřní a vnější stabilita, kriteria.}
\begin{enumerate}[label=(\alph*)]
\item \textit{stabilita}
\begin{enumerate}[label=\alph*/]
\item \textit{vnitřní}: stabilita rovnovážného stavu měřeného systému $ \dot{x}(t) = Ax(t), \, x(t_0), \, x_r = 0 $
\begin{itemize}
\item LDS je stabilní $ \iff \forall \, x_i: Re\{x_i\} < 0 $
\item LDS je asymptoticky stabilní $ \iff \forall x(t_0): \underset{t \to \infty}{\mathrm{lim}} ||x(t)-x_r|| = 0 $
\end{itemize}
\item \textit{vnější}: BIBO: $ y(t) = g(t) = c^T e^{A(t-t_0)} x(t_0) + c^T \displaystyle{\int_{t_0}^t} e^{A(t-\tau)} Bu(\tau) d\tau $
\end{enumerate}
\item \textit{řiditelnost, dosažitelnost stavu}: Matice dosažitelnosti (řiditelnosti): $ Q_D \coloneqq [b, Ab, ..., A^{n-1}b] $. Systém je řiditelný (dosažitelný) $ \iff h[Q_D] = n = dim(x) $.
\vspace{2cm}
\item \textit{pozorovatelnost a rekonstruovatelnost}: "Jsem schopný získat počáteční stav?" Matice pozorovatelnosti: $ Q_P \coloneqq [C, A^T \cdot C, ..., (A^{n-1})^T \cdot C]^T $. \\ Systém je pozorovatelný $ \iff h[Q_P] = n = dim(x) $.
\vspace{2cm}

Kritéria polynomiálně (Hautus):
\begin{itemize}
\item LDS je řiditelný (dosažitelný) $ \iff h[pI-A,b] = n, \, \forall p=\lambda_i, i=1,...,n $ ($ < n $ pro neřiditelný)
\item LDS je pozorovatelný (rekonstruovatelný) $ \iff h[pI-A, C^T]^T = n, \, \forall p=\lambda_i, i=1,...,n $ ($ < n $ pro nepozorovatelný)
\end{itemize}
\item \textit{stabilizovatelnost} (oslabená řiditelnost - nebude řiditelný, ale aspoň stabilní): $ u(x) = -kx(t), \qquad \dot{x}(t) = (A-bk^T) $
\item \textit{detekovatelnost} (oslabená pozorovatelnost): $ u(y) = -hy(t) $
\item \textit{Duální systém}: $ S(A', b, C^T) = S_{dual}(A^T, C, b^T) $. Systém je v minimální realizaci, pokud je řiditelný a zároveň pozorovatelný.
\end{enumerate}
(Kalmanova dekompozice systému)

\subsection{Časové a frekvenční odezvy elementárních členů regulačních obvodů.}

\subsection{Základní typy spojitých a diskrétních regulátorů (P,PI,PID, stavové regulátory a stavové regulátory s integračním charakterem), popis, vlastnosti.}

\subsection{Struktura regulačních obvodů s jedním a dvěma stupni volnosti, přenosy v regulačním obvodu, princip vnitřního modelu.}

\subsection{Problém umístitelnosti pólů a nul nedynamickými a dynamickými regulátory. Požadavky na umístění pólů, konečný počet kroků regulace.}

\subsection{Požadavky na funkci a kvalitu regulace (přesnost regulace, dynamický činitel regulace, kmitavost, robustnost ve stabilitě a j.), omezení na dosažitelnou kvalitu regulace.}

\subsection{Metoda geometrického místa kořenů, pravidla pro konstrukci a využití při syntéze regulátorů, příklady.}

\subsection{Přístup k syntéze regulátorů v klasické teorii regulace, klasické metody, heuristické metody.}

\subsection{Deterministická rekonstrukce stavu, stavový regulátor s rekonstruktorem stavu.}

\subsection{Ljapunovova teorie stability. Ljapunovova rovnice.}

\newpage
\section{Teorie odhadu [TOD]}

\begin{table}[H]
\centering
\begin{tabular}{p{4cm} p{12cm}}
\textit{vyučující:}             & Prof. Ing. Miroslav Šimandl, CSc. \\
								 & Ing. Jindřich Duník, Ph.D. \\
\textit{ročník/semestr studia:} & 3.ročník/ZS \\
\textit{datum zkoušky:}         & 28. 4. 2014 \\
\textit{hodnocení:}             & 1 \\
\textit{cíl předmětu (STAG):}   & \\
\multicolumn{2}{p{16cm}}{Cílem předmětu je obeznámit studenty s možnostmi odhadu parametrů, náhodných veličin a náhodných procesů v podmínkách neurčitosti z apriorních informací a měřených dat.}
\end{tabular}
\end{table}

\subsection{Problémy odhadu, základní etapy vývoje teorie odhadu, náhodné veličiny, náhodné procesy a jejich popis, stochastický systém.}

\newpage
\subsection{Optimální odhad ve smyslu střední kvadratické chyby. Odhad ve smyslu maximální věrohodnosti.}

\newpage
\subsection{Jednorázové a rekurzivní odhady.}

\newpage
\subsection{Odhad stavu lineárního diskrétního systému – filtrace (Kalmanův filtr).}

\newpage
\subsection{Úlohy odhadu stavu lineárního diskrétního stochastického systému – predikce a vyhlazování.}

\newpage
\subsection{Odhad stavu lineárního systému se spojitým či diskrétním měřením (Kalman-Bucyho filtr).}

\newpage
\section{Optimální systémy [OPS]}

\begin{table}[H]
\centering
\begin{tabular}{p{4cm} p{12cm}}
\textit{vyučující:}             & Ing. Miroslav Flídr, Ph.D. \\
								 & Ing. Ivo Punčochář, Ph.D. \\
\textit{ročník/semestr studia:} & 4.ročník/LS \\
\textit{datum zkoušky:}         & 15. 7. 2015 \\
\textit{hodnocení:}             & 3 \\
\textit{cíl předmětu (STAG):}   & \\
\multicolumn{2}{p{16cm}}{Cílem předmětu je seznámení studentů s různými typy optimalizačních úloh. Studenti se naučí řešit jednak základní statické optimalizační úlohy tak především úlohy optimalizace dynamických systémů. Důraz je kladen především na pochopení řešení následujících problémů: 
\begin{itemize}
\item časově optimální řízení;
\item Pontrjaginův princip minima;
\item dynamické programování a Bellmanova optimalizační rekurze;
\item lineárně - kvadratická úloha optimálního řízení.
\end{itemize}}
\end{tabular}
\end{table}

\subsection{Optimální programové řízení diskrétních dynamických systémů. Formulace úlohy. Hamiltonova funkce. Nutné podmínky pro optimální řízení.}
\subsubsection*{Formulace úlohy}
\textit{Dáno:}
\begin{enumerate}
\item Systém: $ x_{k+1} = f_k(x_k, u_k), \qquad k=0,...,N-1, \qquad x_0 $ známé, $ x_k \in R^n, u_k \in R^m $
\item Kritérium: $ J(x_0) = \Phi(x_N) + \displaystyle{\sum_{k=0}^{N-1}}L_k(x_k,u_k) $, kde $ \Phi(x_N) $ je ohodnocení koncového stavu a $ L_k(x_k, u_k) $ je váhová funkce.
\end{enumerate}
\textit{Cílem} je nalézt posloupnost řízení $ u_0^{N-1} $ takovou, aby kritérium nabývalo své minimální hodnoty při splnění vazbových podmínek $ x_{k+1} - f_k(x_k, u_k) = 0 $.

\subsubsection*{Rozšířené kritérium a Hamiltonova funkce}
Diferenční rovnice popisující dynamiku systému jsou chápány jako vazbové podmínky. Definujeme rozšířené kritérium (v podstatě Lagrangeovu funkci):
\begin{equation}
\bar{J}(x_0) = \Phi(x_N) + \displaystyle{\sum_{k=0}^{N-1}}\left[ L_k(x_k,u_k) + \lambda_{k+1}^T(f_k(x_k,u_k)-x_{k+1}) \right]
\end{equation}
kde $ \lambda_1^N $ představuje Lagrangeovy multiplikátory. Při splnění vazbových podmínek nabývá rozšířené kritérium extrémů ve stejných bodech jako původní kritérium , tj. $ \mathrm{min} J(x_0) = \mathrm{min} \bar{J}(x_0) $. Toto kritérium můžeme alternativě vyjádřit ve tvaru:
\begin{equation}
\bar{J}(x_0) = \Phi(x_N) - \lambda_N^T x_N + \displaystyle{\sum_{k=0}^{N-1}}\left[ H_k(x_k,u_k,\lambda_{k+1}) - \lambda_{k+1}^T x_{k+1} \right] + H_0(x_0,u_0,\lambda_1),
\end{equation}
kde funkce $ H_k(x_k,u_k,\lambda_{k+1}) = L_k(x_k,u_k) + \lambda_{k+1}^T f_k(x_k,u_k) $ se nazývá Hamiltonova funkce (Hamiltonián).

\subsubsection*{Nutné podmínky optimálního řízení}
Hledané řešení vyhovuje podmínce $ d\bar{J} = 0 $, kde $ d\bar{J} $ vyjadřuje totální diferenciál vzhledem ke všem veličinám $ x_0^N $, $ u_0^{N-1} $ a $ \lambda_1^N $. "Nulováním" jednotlivých členů definujících $ d\bar{J} $ obdržíme nutné podmínky extrému. Je možné ukázat, že k vyjádření nutných podmínek je potřeba znát jen Hamiltonián. Hamiltonovy kanonické rovnice:
\begin{align*}
\begin{split}
x_{k+1} &= \left(\frac{\partial H_k}{\partial \lambda_{k+1}} \right)^T = f_k(x_k,u_k), \qquad x_0 \\
\lambda_k &= \left(\frac{\partial H_k}{\partial x_k} \right)^T = \left(\frac{\partial L_k}{\partial x_k} \right)^T + \left(\frac{\partial f_k}{\partial x_k} \right)^T \lambda_{k+1}, \qquad \lambda_N = \left(\frac{\partial \Phi}{\partial x_N} \right)^T \\
0 &= \left(\frac{\partial H_k}{\partial u_k} \right)^T = \left(\frac{\partial L_k}{\partial u_k} \right)^T + \left(\frac{\partial f_k}{\partial u_k} \right)^T \lambda_{k+1}
\end{split}
\end{align*}
Řešení představuje dvoubodový okrajový problém, který je v obecném případě obtížně analyticky řešitelný!

\subsubsection*{Modifikace úlohy pro pevný konec}
Předchozí formulace a řešení jsou platné pro úlohu s pevným časem a volným koncem. V případě pevného konce je definována dodatečná sada vazbových podmínek specifikujících koncovou podmínku pro stav ve tvaru $ \Psi(x_N) = 0 $, např. $ \Psi(x_N) = x_N - x_F = 0 $, tj. požadavek na $ x_N = x_F $ pro pevně daný finální stav $ x_f $. Rozšířené kritérium pro tuto úlohu je dáno vztahem:
\begin{equation}
\bar{J}(x_0) = \Phi(x_N) + \nu^T \Psi(x_N) + \displaystyle{\sum_{k=0}^{N-1}}\left[ L_k(x_k,u_k) + \lambda_{k+1}^T(f_k(x_k,u_k)-x_{k+1}) \right],
\end{equation}
kde $ \nu $ mají též význam (dalších) Lagrangeových multiplikátorů. Nutné podmínky optimálního řízení (tj. Hamiltonovy kanonické rovnice) zůstávají nezměněny. Rozšířena
je pouze koncová okrajová podmínka
\begin{equation}
\lambda_N = \left(\frac{\partial \Phi}{\partial x_N} \right)^T + \left(\frac{\partial \Psi}{\partial x_N} \right)^T \nu
\end{equation}

\subsection{Optimální programové řízení spojitých dynamických systémů. Formulace úlohy. Hamiltonova funkce. Nutné podmínky pro optimální řízení. Podmínky transverzality. Pontrjaginův princip minima.}
\subsubsection*{Formulace úlohy s pevným časem}
\textit{Dáno:}
\begin{enumerate}
\item Systém: $ \dot{x}_{t} = f(x(t), u(t), t), \qquad t \in \langle t_0, t_f \rangle, \qquad x(t_0) $ známé, $ x(t) \in R^n, u(t) \in R^m $
\item Kritérium: $ J(x(t_0)) = \Phi(x(t_f), t_f) + \displaystyle{\int_{t_0}^{t_f}} L(x(t),u(t),t) dt $, kde $ \Phi(x(t_f), t_f) $ je ohodnocení koncového stavu a $ L(x(t), u(t), t) $ je váhový funkcionál.
\end{enumerate}
\textit{Cílem} je nalézt funkci řízení $ u(t) $ jako funkci času $ t \in \langle t_0, t_f \rangle $ takovou, aby kritérium nabývalo své minimální hodnoty při splnění vazbových podmínek $ f(x(t), u(t), t) - \dot{x}(t) = 0 $.

\subsubsection*{Rozšířené kritérium a Hamiltonova funkce}
Diferenciální rovnice popisující dynamiku systému jsou chápány jako vazbové podmínky. Definujeme rozšířené kritérium (v podstatě Lagrangeovu funkci):
\begin{equation}
\bar{J}(x(t_0)) = \Phi(x(t_f), t_f) + \displaystyle{\int_{t_0}^{t_f}}\left[ L(x(t),u(t), t) + \lambda^T(t) (f(x(t),u(t), t) - \dot{x}(t)) \right] dt
\end{equation}
kde $ \lambda(t) $ představuje funkci Lagrangeových multiplikátorů. Při splnění vazbových podmínek nabývá rozšířené kritérium extrémů ve stejných bodech jako původní kritérium , tj. $ \mathrm{min} J(x_0) = \mathrm{min} \bar{J}(x_0) $. Toto kritérium můžeme alternativě vyjádřit ve tvaru:
\begin{equation}
\bar{J}(x(t_0)) = \Phi(t(x_f), t_f)) - \lambda^T(t_f) x(t_f) + \lambda^T(t_0) x(t_0) + \displaystyle{\int_{t_0}^{t_f}}\left[ H(x(t),u(t),\lambda(t), t) + \lambda^T(t) x(t) \right] dt,
\end{equation}
kde funkce $ H(x(t), u(t), \lambda(t), t) = L(x(t),u(t), t) + \lambda^T(t) f(x(t), u(t), t) $ se nazývá Hamiltonova funkce (Hamiltonián).

\subsubsection*{Nutné podmínky optimálního řízení}
Hledané řešení vyhovuje podmínce $ \partial \bar{J} = 0 $, kde $ \partial \bar{J} $ vyjadřuje variaci funkcionálu $ \bar{J} $ vzhledem k variacím funkcí $ x(t) $, $ u(t) $ a $ \lambda(t) $. Vhodnou volbou funkce multiplikátorů a s využitím faktu že variace $ \partial x(t_0) = 0 $ je možné se vyhnout vyšetřování závislosti na variaci stavu $ \partial \dot{x}(t) = \frac{\partial f}{\partial x} \partial x + \frac{\partial f}{\partial u} \partial u $ a zároveň obdržíme nutné podmínky optimálního řízení. Je možné ukázat, že k vyjádření nutných podmínek je potřeba znát jen Hamiltonovu funkci. 

Hamiltonovy kanonické rovnice:
\begin{align*}
\begin{split}
\dot{x}(t) &= \left( \frac{\partial H}{\partial \lambda} \right)^T = f(x(t),u(t),t), \qquad x(t_0) \\
\dot{\lambda}(t) &= - \left( \frac{\partial H}{\partial x} \right)^T = - \left( \frac{\partial L}{\partial x} \right)^T - \left( \frac{\partial f}{\partial x} \right)^T \lambda, \qquad \lambda(t_f) = \left( \frac{\partial \Phi(t_f)}{\partial x(t_f)} \right)^T \\
0 &= \left( \frac{\partial H}{\partial u} \right)^T = \left( \frac{\partial L}{\partial u} \right)^T + \left( \frac{\partial f}{\partial u} \right)^T \lambda
\end{split}
\end{align*}
Řešení představuje dvoubodový okrajový problém, který je v obecném případě obtížně analyticky řešitelný!

\subsubsection*{Úloha s volným časem - podmínka transverzality}
Pro úlohu s volným časem je nutné uvažovat i vliv diferenciální změny času $ dt_f $ na variaci rozšířeného kritéria $ \partial \bar{J} $.
\vspace{2.5cm}
Řešení problému je opět dáno dříve uvedenými Hamiltonovými kanonickými rovnicemi. Modifikována je ovšem koncová okrajová podmínka pro určení trajektorie multiplikátorů (tato nová koncová podmínka se nazývá podmínka transverzality):
\begin{equation}
\left[\left(\frac{\partial \Phi}{\partial x} - \lambda^T \right) dx(t_f) + \left(\frac{\partial \Phi}{\partial t} + H \right) dt_f \right]_{t=t_f} = 0
\end{equation}
Speciální případy úlohy a příslušné podmínky transverzality:
\begin{itemize}
\item \textit{Volný konec trajektorie}: Pro libovolné diferenciální změny $ dt_f $ a libovolné $ dx(t_f) $ musí platit:
\begin{equation}
\left( \frac{\partial \Phi}{\partial x} - \lambda^T \right)_{t=t_f} = 0; \qquad \left( \frac{\partial \Phi}{\partial t} + H \right)_{t=t_f} = 0
\end{equation}
\item \textit{Pevný čas a volný konec trajektorie}: Je-li koncový čas určen, je $ dt_f = 0 $ a podmínky transverzality se redukují na:
\begin{equation}
\left( \frac{\partial \Phi}{\partial x} - \lambda^T \right)_{t=t_f} = 0
\end{equation}
\item \textit{Volný čas a pevný konec trajektorie}: Např. úloha časově optimálního řízení. Pokud je koncový stav $ x(t_f) $ pevně určený v libovolném konečném čase $ t_f $, je $ dx(t_f) = 0 $ a podmínky transverzality přejdou do tvaru:
\begin{equation}
\left( \frac{\partial \Phi}{\partial t} + H \right)_{t=t_f} = 0
\end{equation}
\item \textit{Volný čas a konec trajektorie na nadploše}: Je-li koncový bod dán jako bod nadplochy $ \Psi (x(t_f)) = 0 $, pak je tato podmínka chápána jako další vazbová podmínka a podmínka transvezality je dána jako:
\begin{equation}
\left[ \left( \frac{\partial \Phi}{\partial x} - \lambda^T + \frac{\partial \Psi}{\partial x} \nu^T \right) dx(t_f) + \left( \frac{\partial \Phi}{\partial t} + H \right) dt_f \right]_{t=t_f} = 0
\end{equation}
\item \textit{Koncový bod leží na křivce $ x = \Psi(t) $}: V tomto případě nejsou diferenciální změny $ dt_f $ a $ dx(t_f) $ nezávislé, ale zjevně platí $ dx(t_f) = \frac{d \Psi(t)}{dt} dt_f = \dot{\Psi}(t) dt_f $. Dosazením do obecné podmínky transverzality dostaneme:
\begin{equation}
\left[ \left( \frac{\partial \Phi}{\partial x} - \lambda^T \right) \dot{\Psi} \right]_{t=t_f} = \left( \frac{\partial \Phi}{\partial t} + H \right)_{t=t_f}
\end{equation}
Je zřejmé, že pokud je křivka popsaná konstantní funkcí času $ \Psi(t) = konst $, platí $ \dot{\Psi} = 0 $, a dostáváme úlohu s pevným koncem a volným časem.
\end{itemize}

\subsubsection*{Časově optimální řízení}
Pro tuto úlohu je váhový funkcionál a ohodnocení koncového stavu dáno jako $ L = 1$, a $ \Phi = 0 $ a kritérium
optimality má tedy tvar $ J = t_f - t_0 $. Optimální trajektorie je funkcí času a získáme ji vyřešením dvouokrajové úlohy:
\begin{align*}
\begin{split}
\dot{x} &= f(x, u, t), \qquad x(t_0), x(t_f) \\
\dot{\lambda} &= - \left( \frac{\partial f}{\partial x} \right)^T \lambda, \qquad (\lambda^T f)_{t=t_f} = -1 \\
0 &= \left( \frac{\partial f}{\partial u} \right)^T \lambda
\end{split}
\end{align*}

Pokud řešíme \textit{časově invariantní problém}, lze ukázat, že pro t-invariantní systém a kritérium platí, že $ \dot{H}(x^*(t), u^*(t), \lambda^*(t)) = 0 $, tj. H je konstantní podél optimální trajektorie.

\subsubsection*{Pontrjaginův princip minima}
Máme-li dány optimální trajektorie stavu $ x^*(t) $ a multiplikátorů $ \lambda^*(t) $, tj. jsou splněny následující nutné podmínky:
\begin{align*}
\begin{split}
\dot{x}(t) &= \left( \frac{\partial H}{\partial \lambda} \right)^T = f(x(t), u(t), t), \qquad x(t_0) = x_0 \\
\dot{\lambda} &= - \left( \frac{\partial H}{\partial x} \right)^T = - \left( \frac{\partial L}{\partial x} \right)^T - \left( \frac{\partial f}{\partial x} \right)^T \lambda, \qquad \left[ \left( \frac{\partial \Phi}{\partial x} - \lambda^T \right) dx(t_f) + \left( \frac{\partial \Phi}{\partial t} + H \right) dt_f \right]_{t=t_f} = 0
\end{split}
\end{align*}
s počáteční podmínkou $ x(t_0) $ a koncovou okrajovou podmínkou danou obecnou podmínkou transverzality. Pak optimální řízení jako funkce času je hledáno jako
\begin{equation}
\boxed{u^*(t) = \underset{u(t) \in U \subset R^m}{\mathrm{argmin}} H(x^*(t), u(t), \lambda^*(t), t)}
\end{equation}
Toto je nutná podmínka globálního minima a plně nahrazuje podmínku lokálního minima $ \left(\frac{\partial H}{\partial u} \right)^T = 0 $. Největší uplatnění Pontriaginův princip minima přináší pro případy, kdy jsou uvažovány omezení funkce řízení a stavu. V takovém případě platí, že $ \partial \bar{J} \geq 0 $ (protože již není možné libovolně variovat $ \partial u $).

\subsection{Deterministický diskrétní systém automatického řízení. Princip optimality. Bellmanova funkce. Bellmanova optimalizační rekurze.}
\subsubsection*{Formulace úlohy}
\textit{Dáno:}
\begin{enumerate}
\item Systém: $ x_{k+1} = f_k(x_k, u_k), \qquad k=0,...,N-1, \qquad x_0 $ známé, $ x_k \in R^n, u_k \in R^m $
\item Kritérium: $ J(x_0, u_0^{N-1}) = \left[ \Phi(x_N) + \displaystyle{\sum_{k=0}^{N-1}} L_k(x_k,u_k) \right]_{\mathrm{P.T.S}} $, kde $ \Phi(x_N) $ je ohodnocení koncového stavu a $ L_k(x_k, u_k) $ je váhová funkce.
\end{enumerate}
\textit{Cílem} je nalézt posloupnost strategií řízení $ g_0^{N-1} $, které generují posloupnost řízení $ u_0^{N-1} \in U_0^{N-1} $ takovou, aby kritérium nabývalo své minimální hodnoty.

\subsubsection*{Koncová část kritéria optimality - princip optimality}
Koncová část kritéria optimality je definována jako:
\begin{equation}
J_k(x_k, u_k^{N-1}) = \left[ \Phi(x_N) + \displaystyle{\sum_{l=k}^{N-1}} L_l(x_l,u_l) \right]_{\mathrm{P.T.S.}}
\end{equation}
To může být vyjádřeno rekurentním vztahem:
\begin{equation*}
J_k(x_k, u_k^{N-1}) = \left[ L_k(x_k,u_k) + J_{k+1}(x_{k+1},u_{k+1}^{N-1}) \right]_{\mathrm{P.T.S.}} = \left[ L_k(x_k,u_k) + J_{k+1}(f_k(x_k,u_k), u_{k+1}^{N-1}) \right]_{\mathrm{P.T.S.}}
\end{equation*}
Mějme optimální posloupnost strategií řízení $ g_0^{N-1} $ generující optimální posloupnost řízení $ u_0^{N-1} $. Rozdělíme-li trajektorii v čase $ k $ na neměnnou minulost a neuskutečněnou budoucnost, pak bude platit, že:
\begin{equation*}
J^*(x_0) = J_0(x_0, u_0^{* N-1}) + \left[ \displaystyle{\sum_{l=0}^{k-1}}(L_l(x_l^*, u_l^*)) + J_k(x_k^*,u_k^{* N-1}) \right]_{\mathrm{P.T.S.}}
\end{equation*}
\textbf{Princip optimality}: \textit{Každý koncový úsek optimální trajektorie je optimálním koncovým úsekem trajektorie.}

\subsubsection*{Bellmanova funkce}
Bellmanova funkce je definována jako minimální hodnota koncového úseku kritéria:
\begin{equation}
V_k(x_k) = \underset{u_k^{N-1} \in U_k^{N-1}}{\mathrm{min}} J_k(x_k, u_k^{N-1})
\end{equation}
S využitím rekurentního vztahu pro koncovou část kritéria lze Bellmanovu funkci vyjádřit jako:
\begin{equation}
V_k(x_k) = \underset{u_k \in U_k}{\mathrm{min}} \left[ L_k(x_k, u_k) + \underset{u_{k+1}^{N-1} \in U_{k+1}^{N-1}}{\mathrm{min}} J_{k+1}(x_{k+1}, u_{k+1}^{N-1}) \right]
\end{equation}

\subsubsection*{Bellmanova optimalizační rekurze}
\begin{align*}
\begin{split}
V_k(x_k) &= \underset{u_k \in U_k}{\mathrm{min}} \left[ L_k(x_k, u_k) + V_{k+1}(f_k(x_k,u_k)) \right], \qquad k= N-1, N-2, ..., 0 \\
u_k &= g_k(x_k) = \underset{u_k \in U_k}{\mathrm{argmin}} \left[ L_k(x_k, u_k) + V_{k+1}(f_k(x_k,u_k)) \right], \qquad k= N-1, N-2, ..., 0 \\
V_N(x_N) &= J_N(x_N) = \Phi(x_N)
\end{split}
\end{align*}
Minimální hodnota kritéria je dána jako $ J^*(x_0) = J_0(x_0,u_0^{* N-1}) = V_0(x_0) $

\subsection{Syntéza optimálního deterministického systému automatického řízení pro diskrétní lineární řízený systém a kvadratické kritérium. Formulace a řešení. Asymptotické řešení a jeho stabilita.}
\subsubsection*{Formulace úlohy}
\textit{Dáno:}
\begin{enumerate}
\item Systém: $ x_{k+1} = A_k x_k + B_k u_k, \qquad k=0,...,N-1, \qquad x_0 $ známé, $ x_k \in R^n, u_k \in R^m $
\item Kritérium: $ J(x_0, u_0^{N-1}) = \left[ x_N^T Q_N x_N + \displaystyle{\sum_{k=0}^{N-1}} \begin{bmatrix} x_k \\ u_k \end{bmatrix}^T \begin{bmatrix} Q_k & S_k \\ S_k^T & R_k \end{bmatrix} \begin{bmatrix} x_k \\ u_k \end{bmatrix} \right]_{\mathrm{P.T.S}} $, 

kde je $ \begin{bmatrix} Q_k & S_k \\ S_k^T & R_k \end{bmatrix} = \begin{bmatrix} Q_k & S_k \\ S_k^T & R_k \end{bmatrix}^T, \qquad Q_k \geq 0, R_k > 0 $.
\end{enumerate}
\textit{Cílem} je nalézt posloupnost strategií řízení $ g_k(x_k) \, \forall k $, které generují posloupnost řízení $ u_0^{N-1} \in U_0^{N-1} $ takovou, aby kritérium nabývalo své minimální hodnoty.

\subsubsection*{Bellmanova funkce}
Indukcí lze dokázat, že Bellmanova funkce je dána kvadratickou formou $ V_k(x_k) = x_k^T P_k x_k \, \forall k $, kde $ P_k = P_k^T $ je pozitivně semidefinitní matice. Pro koncový čas pak platí $ V_N(x_N) = x_N^T Q_N x_N $, z čehož vyplývá, že $ P_N = Q_N $.

\subsubsection*{Bellmanova optimalizační rekurze}
Předpokládáme-li, že v čase $ k+1 $ je Bellmanova funkce dána kvadratickou formou $ V_{k+1}(x_{k+1}) = x_{k+1}^T P_{k+1} x_{k+1} $, pak je optimalizační rekurze dána vztahy:
\begin{align*}
\begin{split}
V_k(x_k) &= \underset{u_k \in U_k}{\mathrm{min}} \left[ \begin{bmatrix} x_k \\ u_k \end{bmatrix}^T \begin{bmatrix} Q_k & S_k \\ S_k^T & R_k \end{bmatrix} \begin{bmatrix} x_k \\ u_k \end{bmatrix} + V_{k+1}(A_k x_k + B_k u_k) \right], \qquad k= N-1, N-2, ..., 0 \\
u_k &= g_k(x_k) = \underset{u_k \in U_k}{\mathrm{argmin}} \left[ \begin{bmatrix} x_k \\ u_k \end{bmatrix}^T \begin{bmatrix} Q_k & S_k \\ S_k^T & R_k \end{bmatrix} \begin{bmatrix} x_k \\ u_k \end{bmatrix} + V_{k+1}(A_k x_k + B_k u_k) \right], \qquad k= N-1, N-2, ..., 0 \\
V_N(x_N) &= x_N^T Q_N x_N
\end{split}
\end{align*}

\subsubsection*{Řešení lineárně-kvadratické úlohy (LQ úlohy)}
\begin{enumerate}
\item Úloha má vždy řešení a toto řešení je jediné.
\item Bellmanova funkce je pozitivně semidefinitní kvadratická forma $ V_k(x_k) = x_k^T P_k x_k \, \forall k, \, P_k = P_k^T \geq 0 $
\item Řešení je dáno pro okamžiky $ k = N-1,...,0 $ řešením \textit{Riccatiho diferenční rovnice}:
\begin{equation}
P_k = A_k^T P_{k+1} A_k + Q_k - (A_k^T P_{k+1} B_k + S_k)(B_k^T P_{k+1} B_k + R_k)^{-1}(A_k^T P_{k+1} B_k + S_k)^T
\end{equation}
s koncovou okrajovou podmínkou $ P_N = Q_N $.
\item Optimální zpětnovazební strategie řízení při měřeném stavu je:
\begin{equation*}
u_k(x_k) = g_k(x_k) = -(B_k^T P_{k+1} B_k + R_k)^{-1}(A_k^T P_{k+1} B_k + S_k)^T x_k = - K_k x_k
\end{equation*}
\item Hodnota kritéria optimality podél trajektorie optimálního (uzavřeného) systému je:
\begin{equation*}
J^*(x_0) = x_0^T P_0 x_0
\end{equation*}
\end{enumerate}

\subsubsection*{Asymptotická LQ úloha}
\begin{itemize}
\item Uvažujeme t-invariantní systém a nekonečný časový horizont:
\begin{equation}
x_{k+1} = A_k x_k + B_k u_k, \qquad k=0,1,...
\end{equation}
kde dvojice $ (A, B) $ je řiditelná a počátek je rovnovážným stavem, tj. $ f(0, 0) = 0 $.
\item Kritérium je též uvažováno jako t-invariantní:
\begin{equation}
J(x_0, u_0^{\infty}) = \left[ \displaystyle{\sum_{k=0}^{\infty}} \begin{bmatrix} x_k \\ u_k \end{bmatrix}^T \begin{bmatrix} Q & S \\ S^T & R \end{bmatrix} \begin{bmatrix} x_k \\ u_k \end{bmatrix} \right]_{\mathrm{P.T.S}}
\end{equation}
Pro rovnovážný stav v počátku platí, že $ L(0, 0) = 0 $ a tedy, že i při nekonečném horizontu řízení nabývá kritérium konečné hodnoty.
\end{itemize}
Pro řešení asymptotické LQ úlohy platí:
\begin{enumerate}
\item Bellmanova funkce je pozitivně semidefinitní \textit{časově invariantní} kvadratická forma $ V_k(x_k) = x_k^T P x_k \, \forall k, \, P = P^T \geq 0 $
\item Matice $ P $ definující kvadratickou formu Bellmanovy funkce je dáno řešením \textit{Riccatiho diferenční rovnice}:
\begin{equation}
P = A^T P A + Q - (A^T P B + S)(B^T P B + R)^{-1}(A^T P B + S)^T
\end{equation}
\item Optimální zpětnovazební strategie řízení při měřeném stavu je též \textit{časově invariantní}:
\begin{equation*}
u_k(x_k) = g(x_k) = -(B^T P B + R)^{-1}(A^T P B + S)^T x_k = - K x_k
\end{equation*}
\item Hodnota kritéria optimality podél trajektorie optimálního (uzavřeného) systému je:
\begin{equation*}
J^*(x_0) = x_0^T P x_0
\end{equation*}
\end{enumerate}
Podmínky stability asymptotického řešení LQ úlohy:
\begin{itemize}
\item Je-li Bellmanova funkce Ljapunovovou funkcí, tj. řešení Riccatiho algebraické rovnice $ P $ je pozitivně definitní, pak je rovnovážný stav v počátku ($ x = 0 $) globálně asymptoticky stabilní.
\item Je-li dvojice $ (A, B) $ řiditelná a dvojice $ (C, \bar{A}) $ pozorovatelná, pak je rovnovážný stav globálně asymptoticky stabilní. Matice $ C $ a  $ \bar{A} $ jsou definovány následovně:
\begin{equation*}
C^TC = Q - SR^{-1}S^T; \qquad \bar{A} = A + BR^{-1}S^T
\end{equation*}
Poznámky:
\begin{itemize}
\item Zavádí se fiktivní pozorování $ z_k = Cx_k $ skrze kritérium $ J(x_0, u_0^\infty = \displaystyle{\sum_{k=0}^\infty} z_k^T z_k + u_k^T R u_k $.
\item Je-li tato podmínka splněna je matice $ P $ zároveň pozitivně definitní.
\item Stabilita je zaručena i pro případ detekovatelné dvojice $ (C, \bar{A}) $, tj. jsou-li nepozorovatelné póly stabilní, a matice $ P $ je pak pozitivně semidefinitní.
\end{itemize}
\end{itemize}

\subsection{Deterministický spojitý systém automatického řízení. Kontinualizace Bellmanovy optimalizační rekurze.}
\subsubsection*{Formulace úlohy}
\textit{Dáno:}
\begin{enumerate}
\item Systém: $ \dot{x}(t) = f(x(t), u(t), t), \qquad t \in \langle t_0, t_f \rangle, \qquad x(t_0) $ známé, $ x(t) \in R^n, u(t) \in R^m $
\item Kritérium: $ J(x(t_0), u_{t_0}^{t_f}) = \left[ \Phi(x(t_f), t_f) + \displaystyle{\int_{t_0}^{t_f}} L(x(t), u(t), t) dt \right]_{\mathrm{P.T.S}} $, kde $ \Phi(x(t_f), t_f) $ je ohodnocení koncového stavu a $ L(x(t), u(t), t) $ je váhový funkcionál.
\end{enumerate}
\textit{Cílem} je nalézt funkci strategií řízení $ g(x(t), t) $, pro $ t \in \langle t_0, t_f \rangle $, která generuje řízení $ u(t) $ pro $ t \in \langle t_0, t_f \rangle $ takové, aby kritérium nabývalo své minimální hodnoty.

\subsubsection*{Využití známého řešení pro diskrétní systém pomocí kontinualizace}
Řešení hledáme pro diskrétní systém: $ x(t_k + h) \approx x(t_k) + f(x(t_k), u(t_k), t_k) \cdot h $, kde $ h = \frac{(t_f - t_0)}{N} $ a $ t_k = t_0 + kh, \, k=0,...,N-1 $. Kritérium kvality uvažujeme ve tvaru:
\begin{equation}
J(x_0, u_{t_0}^{t_f}) \approx \left[ \Phi(x(t_N), t_N) + \displaystyle{\sum_{k=0}^{N-1} L(x(t_k), u(t_k), t_k) h} \right]
\end{equation}
Bellmanovu optimalizační rekurzi pro tento problém vyjádříme následovně:
\begin{align*}
\begin{split}
V(x(t), t) &= \underset{u(t) \in U}{\mathrm{min}} \left[ L(x(t), u(t), t) h + V(x(t+h), t+h) \right]_{\mathrm{P.T.S.}} \\
g(x(t), t) &= \underset{u(t) \in U}{\mathrm{argmin}} \left[ L(x(t), u(t), t) h + V(x(t+h), t+h) \right]_{\mathrm{P.T.S.}} \\
\end{split}
\end{align*}
s okrajovou podmínkou $ V(x(t_f), t_f) = \Phi(x(t_f), t_f) $. Problém pak řešíme pro $ h \to 0+ $, tj. kontinualizací diskrétního řešení.

\subsubsection*{Bellmanova funkce pro spojitý deterministický systém}
Je dána ve formě parciální diferenciální rovnice:
\begin{equation}
- \frac{\partial V(x,t)}{\partial t} = \underset{u(t) \in U}{\mathrm{min}} \left[ L(x, u, t) + \frac{\partial V(x,t)}{\partial x} f(x,u,t) \right]
\end{equation}
s okrajovou podmínkou $ V(x(t_f), t_f) = \Phi(x(t_f), t_f) $. Tato parciální diferenciální rovnice je obvykle nazývána Hamiltonova-Belmanova-Jacobiho parciální
diferenciální rovnice. \textit{Strategie řízení} je pak hledána jako:
\begin{equation}
g(x,t) = \underset{u(t) \in U}{\mathrm{argmin}} \left[ L(x, u, t) + \frac{\partial V(x,t)}{\partial x} f(x,u,t) \right]
\end{equation}

\subsubsection*{Asymptotická úloha optimálního řízení pro spojitý systém}
Předpoklady pro asymptotickou úlohu:
\begin{enumerate}
\item Uvažujeme t-invariantní systém a nekonečný časový horizont:
\begin{equation*}
\dot{x}(t) = f(x(t), u(t)), \qquad t = \langle 0, \infty \rangle, \qquad f(0,0) = 0
\end{equation*}
\item Kritérium je též uvažováno jako t-invariantní:
\begin{equation*}
J(x(t_0), u_{t_0}^\infty) = \left[ \displaystyle{\int_{t_0}^\infty L(x(t),u(t)) dt} \right]_{\mathrm{P.T.S.}}, \qquad L(x(t), u(t)) \geq 0 \, \forall t \land L(0,0) = 0
\end{equation*}
\end{enumerate}
Bellmanova funkce a strategie řízení (t-invariantní, tj. $ \frac{\partial V}{\partial t} = 0 $):
\begin{align*}
\begin{split}
0 &= \underset{u \in U}{\mathrm{min}} \left[ L(x, u, t) + \frac{\partial V(x)}{\partial x} f(x,u) \right] \\
g(x) &= \underset{u \in U}{\mathrm{argmin}} \left[ L(x, u, t) + \frac{\partial V(x)}{\partial x} f(x,u) \right]
\end{split}
\end{align*}

\subsubsection*{Asymptotická LQ úloha pro spojitý deterministický systém}
\begin{enumerate}
\item Úloha má řešení a to jediné.
\item Matice $ P $ Bellmanovy funkce je časově invariantní symetrická pozitivně semidefinitní matice, která je řešením \textit{Riccatiho diferenční rovnice}:
\begin{equation}
0 = P A + A^T P - (P B + S) R^{-1} (P B + S)^T + Q
\end{equation}
\item Je-li navíc pozorovatelná dvojice $ (\bar{A}, C) $, kde:
\begin{equation*}
C^TC = Q - SR^{-1}S^T; \qquad \bar{A} = A + BR^{-1}S^T
\end{equation*}
je optimální systém \textit{globálně asymptoticky stabilní} a matice $ P $ pozitivně definitní.
\item Hodnota kritéria podél optimální trajektorie je:
\begin{equation*}
J^*(x_0) = x_0^T P x_0
\end{equation*}
\end{enumerate}

\subsection{Optimální stochastický systém automatického řízení. Strategie řízení. Bellmanova funkce a Bellmanova optimalizační rekurze.}
\subsubsection*{Formulace úlohy}
\textit{Dáno:}
\begin{enumerate}
\item Systém: $ \varphi_{k}(x_k, y_k, x_{k-1}, y_{k-1}, u_{k-1}) $, kde $ \begin{bmatrix} x_k \\ y_k \end{bmatrix} \in R^n $ představuje stav, dále $ u_k \in U_k \subset R^m $ a $ \varphi_0(x_0, y_0, 0) $ představuje počáteční podmínku.
\item Kritérium:
\begin{align*}
\begin{split}
J(\gamma_0^N) &= E \left[ L(x_0^N, y_0^N, u_0^N), \gamma_0^N \right] = \int \int \int L(x_0^N, y_0^N, u_0^N) p(x_0^N, y_0^N, u_0^N, \gamma_0^N) dx_0^N dy_0^N du_0^N \\
L(x_0^N, y_0^N, u_0^N) & = \displaystyle{\sum_{k=0}^{N-1}} L_k(x_k, u_k), \gamma_0^N
\end{split}
\end{align*}
\end{enumerate}
\textit{Cílem} je nalézt posloupnost strategií řízení $ \gamma_k(y_0^k, u_0^{k-1}) \in \Gamma_k^N $, která pro $ k=0,...,N-1 $ generuje řízení $ u_k $ takové, aby kritérium nabývalo své minimální hodnoty.

\subsubsection*{Kritérium optimality koncového úseku strategie řízení}
\begin{equation*}
J_k(\gamma_k^N, y_0^k, u_0^{k-1} = E \left[ \displaystyle{\sum_{l=k}^N} L_l(x_l, y_l, u_l), y_0^k, u_0^{k-1}, \gamma_k^N \right]
\end{equation*}
které může být vyjádřeno rekurentním vztahem
\begin{equation*}
J_k(\gamma_k^N, y_0^k, u_0^{k-1} = \displaystyle{\int_\infty^\infty} E \left[ L_k(x_k, y_k, u_k) + J_{k+1}(\gamma_{k+1}^N, y_0^{k+1}, u_0^{k}), y_0^k, u_0^k \right]  , \gamma_k(u_k, y_0^k, u_0^{k-1}) du_k
\end{equation*}
V případě deterministické strategie řízení $ \gamma_k^N = \delta (u_k - g_k (y_0^k, u_0^{k-1})) $ je kritériem optimality koncového úseku strategie řízení daného vztahem:
\begin{equation*}
J_k(\gamma_k^N, y_0^k, u_0^{k-1} = E \left[ L_k(x_k, y_k, u_k) + J_{k+1}(\gamma_{k+1}^N, y_0^{k+1}, u_0^{k}), y_0^k, u_0^k \right]
\end{equation*}
Okrajová podmínka je v obou případech $ J_{N+1} = 0 $. Hodnota kritéria podél optimální trajektorie systému je definována jako $ J(\gamma_0^N) = E \left[ J_0(\gamma_0^N, y_0) \right] $.

\subsubsection*{Bellmanova funkce}
Bellmanova funkce je definována jako minimální hodnota koncového úseku kritéria:
\begin{align*}
\begin{split}
V_k(y_0^k, u_0^{k-1}) &= \underset{\gamma_k^N \in \gamma_k^N}{\mathrm{min}} J_k(\gamma_k^N, y_0^k, u_0^{k-1}) \\
\gamma_0^N &= \underset{\gamma_k^N \in \gamma_k^N}{\mathrm{argmin}} J_k(\gamma_k^N, y_0^k, u_0^{k-1})
\end{split}
\end{align*}
Optimalizace je opět založena na rekurzivním výpočtu hodnoty Bellmanovy funkce.

\subsubsection*{Bellmanova optimalizační rekurze}
\begin{align*}
\begin{split}
V_k(y_0^k, u_0^{k-1}) &= \underset{u_k \in u_k}{\mathrm{min}} E \left[ L_k(x_k, y_k, u_k) + V_{k+1}(y_0^{k+1}, u_0^k), y_0^k, u_0^k \right] \\
g_k(y_0^k, u_0^{k-1}) &= \underset{u_k \in u_k}{\mathrm{argmin}} E \left[ L_k(x_k, y_k, u_k) + V_{k+1}(y_0^{k+1}, u_0^k), y_0^k, u_0^k \right] \\
\gamma_0^*(u_k, y_0^k, u_0^{k-1}) &= \delta (u_k - g_k(y_0^k, u_0^{k-1}))
\end{split}
\end{align*}
Okrajová podmínka rekurze je $ V_{N+1}(y_0^N, u_0^N) = 0 $ a hodnota kritéria optimality pro optimální strategii je $ J^*(\gamma_0^{* N}) = E [V_0(y_0)] $.

\subsection{Syntéza optimálního systému automatického řízení pro lineární gaussovský řízený systém a kvadratické kritérium. Formulace a řešení. Separační teorém.}
\subsubsection*{Formulace problému}
\textit{Dáno:}
\begin{enumerate}
\item Systém: 
\begin{align*}
\begin{split}
\varphi_{k}(x_k : x_{k-1}, u_{k-1}) &= \mathscr{N}(A_{k-1}x_{k-1} + B_{k-1} u_{k-1}, G_{k-1}G_{k-1}^T \\
\varphi_0(x_0: 0) &= \mathscr{N}(\hat{x}_0, \Pi_0) \\
\Psi_k(y_k: x_k &= \mathscr{N}(C_kx_k, H_kH_k^T)
\end{split}
\end{align*}
\item Kritérium:
\begin{align*}
J(g_0^{N-1}) &= E \left[ x_N^T Q_N x_N + \displaystyle{\sum_{k=0}^{N-1}} \begin{bmatrix} x_k \\ u_k \end{bmatrix}^T \begin{bmatrix} Q_k & S_k \\ S_k^T & R_k \end{bmatrix} \begin{bmatrix} x_k \\ u_k \end{bmatrix}; g_0^{N-1} \right]
\end{align*}
\end{enumerate}
\textit{Cílem} je nalézt posloupnost strategií řízení $ g_k(x_k) \, \forall k $, které generují posloupnost řízení $ u_0^{N-1} \in U_0^{N-1} $ takovou, aby kritérium nabývalo své minimální hodnoty.

\subsubsection*{Alternativní ekvivalentní popisy systému}
Popis pomocí stochastické rovnice s gaussovskými normovanými šumy:
\begin{align*}
\begin{split}
x_{k+1} &= A_k x_k + B_k u_k + G_k \xi_k \\
y_k &= C_k x_k + H_k \eta_k \\
p(x_0) &= \mathscr{N}(\hat{x}_0, \Pi_0)
\end{split}
\end{align*}
kde $ \xi_k \in R^n $ a $ \eta_k \in R^p $ jsou navzájem nezávislé gaussovské normované absolutně náhodné diskrétní procesy s rozložením $ \mathscr{N}(0, I) $. Oba tyto procesy jsou též nezávislé s náhodným počátečním stavem $ x_0 $.

Popis pomocí stochastické rovnice s gaussovskými nenormovanými šumy:
\begin{align*}
\begin{split}
x_{k+1} &= A_k x_k + B_k u_k + w_k \\
y_k &= C_k x_k + v_k \\
p(x_0) &= \mathscr{N}(\hat{x}_0, \Pi_0)
\end{split}
\end{align*}
kde $ w_k \in R^n $ a $ v_k \in R^p $ jsou navzájem nezávislé gaussovské absolutně náhodné diskrétní procesy s rozložením $ \mathscr{N}(0, G_{k-1}G_{k-1}^T) $ a $ \mathscr{N}(0, H_{k-1}H_{k-1}^T) $. Oba tyto procesy jsou opět nezávislé s náhodným počátečním stavem $ x_0 $.

\subsubsection*{Kalmanův filtr}
Stav systému $ x_k $ není přímo měřitelný a proto je nutné nalézt jeho odhad. Pro lineární gaussovský systém poskytuje optimální odhad stavu Kalmanův filtr.

Filtrační hustota pravděpodobnosti:
\begin{align*}
\begin{split}
p(x_k | y_0^k, u_0^k) &= \mathscr{N}(u_k(y_0^k, u_0^{k-1}), \Pi_{k|k}) \\
\mu_k &= \hat{x}_k + K_k^F(y_k-C_k\hat{x}_k) \\
K_k^F &= \Pi_{k|k-1}C_k^T(H_kH_k^T + C_k\Pi_{k|k-1}C_k^T) \\
\Pi_{k|k} &= \Pi_{k|k-1} - K_k^FC_k\Pi_{k|k-1}
\end{split}
\end{align*}

Prediktivní hustota pravděpodobnosti:
\begin{align*}
\begin{split}
p(x_{k+1} | y_0^k, u_0^k) &= \mathscr{N}(\hat{x}_{k+1}(y_0^k, u_0^{k-1}), \Pi_{k+1|k}) \\
\hat{x}_{k+1} &= A_k \mu_k + B_k u_k \\
\Pi_{k+1|k} &= A_k\Pi_{k|k}A_k^T + G_kG_k^T
\end{split}
\end{align*}

Počáteční podmínka: $ p(x_0) = \mathscr{N}(\hat{x}_0, \Pi_0) $).

\subsubsection*{Bellmanova funkce}
Indukcí lze dokázat, že Bellmanova funkce je dána kvadratickou formou:
\begin{equation*}
V_k(y_0^k, u_0^{k-1} = \mu_k^T(y_0^k, u_0^{k-1})P_k\mu_k(y_0^k,u_0^{k-1}) + K_k, \, \forall k
\end{equation*}
kde $ P_k = P_k^T $ je pozitivně semidefinitní matice. Pro koncový čas pak platí: 
\begin{equation*}
V_N(y_0^N, u_0^{N-1}) = \mu_N^TP_N\mu_N + K_N
\end{equation*}

\subsubsection*{Bellmanova optimalizační rekurze}
Pro časové okamžiky $ k=N-1,...,0 $ platí rekurentní vztahy, po jejichž úpravě dostaneme Bellmanovu funkci a zákon řízení ve tvaru:
\begin{align*}
\begin{split}
V_k(y_0^k, u_0^{k-1}) &= \underset{u_k \in u_k}{\mathrm{min}} \left[ \begin{bmatrix} u_k \\ u_k \end{bmatrix}^T \begin{bmatrix} A_k^T P_{k+1} A_k + Q_k & A_k^T P_{k+1} B_k + S_k \\ B_k^T P_{k+1} A_k + S_k^T & B_k^T P_{k+1} B_k + R_k \end{bmatrix} \begin{bmatrix} u_k \\ u_k \end{bmatrix} \right] + K_k \\
g_k(y_0^k, u_0^{k-1}) &= \underset{u_k \in u_k}{\mathrm{argmin}} \left[ \begin{bmatrix} u_k \\ u_k \end{bmatrix}^T \begin{bmatrix} A_k^T P_{k+1} A_k + Q_k & A_k^T P_{k+1} B_k + S_k \\ B_k^T P_{k+1} A_k + S_k^T & B_k^T P_{k+1} B_k + R_k \end{bmatrix} \begin{bmatrix} u_k \\ u_k \end{bmatrix} \right]
\end{split}
\end{align*}
Je zřejmé, že $ V_k(y_0^k, u_0^{k-1}) = V_k(\mu_k) $ a $ g_k(y_0^k, u_0^{k-1}) = g_k(\mu_k) $.

\subsubsection*{Separační teorém}
Porovnáním s řešením deterministické LQ úlohy optimálního řízení zjistíme, že platí tzv. Separační teorém: \textit{Řešení LQG úlohy lze rozdělit na řešení dvou úloh a to:}
\begin{enumerate}
\item Návrh optimálního stavového regulátoru pro ekvivalentně určitý (deterministický) systém.
\item Návrh optimálního estimátoru stavu generujícího jeho střední hodnotu podmíněnou dostupným pozorováním.
\end{enumerate}
Tj. v zákonu řízení nahradíme skutečné hodnoty stavu v regulátoru jejich odhady.

\subsubsection*{Řešení LQG úlohy}
\vspace{5cm}

\newpage
\section{Adaptivní systémy [AS]}

\begin{table}[H]
\centering
\begin{tabular}{p{4cm} p{12cm}}
\textit{vyučující:}             & Ing. Jindřich Duník, Ph.D. \\
								 & Ing. Ladislav Král, Ph.D. \\
\textit{ročník/semestr studia:} & 5.ročník/ZS \\
\textit{datum zkoušky:}         & 12. 12. 2016 \\
\textit{hodnocení:}             & 1 \\
\textit{cíl předmětu (STAG):}   & \\
\multicolumn{2}{p{16cm}}{Cílem předmětu je obeznámit studenty s adaptivními systémy automatického řízení a adaptivními systémy zpracování signálů.}
\end{tabular}
\end{table}

\subsection{Základní přístupy k syntéze adaptivních řídicích systémů, schematické vyjádření, srovnání s předpoklady a návrhem standardních regulátorů.}

\newpage
\subsection{Adaptivní řízení s referenčním modelem, MIT pravidlo, využití Ljapunovovy teorie stability.}

\newpage
\subsection{Samonastavující se regulátory, charakteristika a základní přístupy k návrhu bloku řízení, přiřazení pólů, diofantické rovnice, minimální variance.}

\newpage
\subsection{Samonastavující se regulátory, charakteristika a základní přístupy k návrhu bloku poznávání, parametrické metody odhadu.}

\newpage
\subsection{Adaptivní systémy na zpracování signálu, adaptivní prediktor, adaptivní filtr, analogie se samonastavujícími se regulátory.}

\newpage
\subsection{Adaptivní řízení a strukturální vlastnost stochastického optimálního řízení, duální řízení, neutralita, separabilita, ekvivalence určitosti.}