\chapter{Teorie řízení [TŘSZ]}

\section{Lineární systémy 1-2 [LS1], [LS2]}

\begin{table}[H]
\centering
\begin{tabular}{p{4cm} p{12cm}}
\textit{vyučující:}             & Doc. Ing. Jiří Melichar, CSc. \\
								 & Ing. Martin Čech, Ph.D. \\
								 & Ing. Jiří Mertl, Ph.D. \\
\textit{ročník/semestr studia:} & 2.ročník/ZS-LS \\
\textit{datum zkoušky:}         & X. 1. 2013/X. X. 2013 \\
\textit{hodnocení:}             & 1/2 \\
\textit{cíl předmětu (STAG):}   & \\
\multicolumn{2}{p{16cm}}{LS1: Student by měl získat přehled o typech, struktuře a chování reálných dynamických systémů, obeznámit se s metodikou tvorby matematických modelů reálných dynamických systémů a s metodami analýzy jejich vlastností a chování v časové i frekvenční oblasti. Student by měl také porozumět základním principům řízení dynamických systémů a metodám pro získávání potřebných dat z reálných procesů. 

Cílem předmětu LS2 je, aby student:
\begin{itemize}
\item získal přehled o klasických regulačních úlohách, o struktuře regulačních obvodů a o základních typech dynamických i nedynamických regulátorů;
\item dokázal analyzovat reálnou regulační úlohu v její celistvosti, uměl formulovat požadavky na kvalitu regulace v časové i frekvenční oblasti při současném respektování všech omezení;
\item byl schopen použít vhodné metody pro návrh spojitých i číslicových regulátorů a získávat potřebná data z reálného procesu;
\item byl schopen analýzy nelineárních dynamických systémů a základní orientace v problémech jejich řízení.
\end{itemize}}
\end{tabular}
\end{table}

\subsection{Matematické modely spojitých a diskrétních lineárních dynamických systémů.}
Za model považujeme \textit{stavovou reprezentaci} dynamických systémů a nebo dynamiku vyjádřenou \textit{diferenciálními rovnicemi}. K modelu dojdeme buď \textit{experimentálně} a nebo (raději) na základě \textit{matematicko-fyzikálního} modelování (znalost fyzikálních zákonů z dané vědní oblasti)
\begin{itemize}
\item \textit{klasická mechanika}
\begin{enumerate}[label=(\alph*)]
\item Newtonovská mechanika - silové zákony, rovnováha sil
\item Lagrangeova mechanika - znalost kinetické a potenciální energie + elektromechanická analogie
\end{enumerate}
\item \textit{elektrické obvody} - Kirchhoffovy zákony
\item \textit{hydrodynamika} - rovnice kontinuity, Bernoulliho zákon
\item jiné získání struktury a parametrů modelu
\end{itemize}

\subsubsection*{Newtonova mechanika}
Newtonův zákon síly:
\begin{equation}
\boxed{F = m \cdot a}
\end{equation}
K sestavení stavového modelu se zavedou složky vektoru stavu $ x_1(t), x_2(t) $:
\begin{align}
\begin{split}
y(t) &\coloneqq x_1(t) \\
\dot{y}(t) &= \dot{x}_1(t) \coloneqq x_2(t) \\
\end{split}
\end{align}
Víme, že $ x_1(t) $ odpovídá poloze. Derivace polohy, tedy $ x_2(t) $, odpovídá rychlosti. Zrychlení poté odpovídá derivace rychlosti, tedy $ \dot{x}_2(t) = a = \frac{F(t)}{m} $. Nyní už můžeme sestavit stavový model založený na zákonu síly:
\begin{align}
\begin{split}
\dot{x}(t) &= A \cdot x(t) + b \cdot F(t) = \begin{bmatrix}
0 & 1 \\ 0 & 0
\end{bmatrix} \cdot x(t) + \begin{bmatrix}
0 \\ \frac{1}{m}
\end{bmatrix} \cdot F(t) \\
y(t) &= C^T \cdot x(t) = \begin{bmatrix}
1 & 0
\end{bmatrix} \cdot x(t)
\end{split}
\end{align}

\subsubsection*{Lagrangeova mechanika}


\subsection{Linearizace nelineárních dynamických systémů, rovnovážné stavy. Harmonická linearizace.}

\subsection{Vlastnosti lineárních dynamických systémů. Řiditelnost, pozorovatelnost, kriteria. Vnitřní a vnější stabilita, kriteria.}

\subsection{Časové a frekvenční odezvy elementárních členů regulačních obvodů.}

\subsection{Základní typy spojitých a diskrétních regulátorů (P,PI,PID, stavové regulátory a stavové regulátory s integračním charakterem), popis, vlastnosti.}

\subsection{Struktura regulačních obvodů s jedním a dvěma stupni volnosti, přenosy v regulačním obvodu, princip vnitřního modelu.}

\subsection{Problém umístitelnosti pólů a nul nedynamickými a dynamickými regulátory. Požadavky na umístění pólů, konečný počet kroků regulace.}

\subsection{Požadavky na funkci a kvalitu regulace (přesnost regulace, dynamický činitel regulace, kmitavost, robustnost ve stabilitě a j.), omezení na dosažitelnou kvalitu regulace.}

\subsection{Metoda geometrického místa kořenů, pravidla pro konstrukci a využití při syntéze regulátorů, příklady.}

\subsection{Přístup k syntéze regulátorů v klasické teorii regulace, klasické metody, heuristické metody.}

\subsection{Deterministická rekonstrukce stavu, stavový regulátor s rekonstruktorem stavu.}

\subsection{Ljapunovova teorie stability. Ljapunovova rovnice.}

\section{Teorie odhadu [TOD]}

\begin{table}[H]
\centering
\begin{tabular}{p{4cm} p{12cm}}
\textit{vyučující:}             & Prof. Ing. Miroslav Šimandl, CSc. \\
								 & Ing. Jindřich Duník, Ph.D. \\
\textit{ročník/semestr studia:} & 3.ročník/ZS \\
\textit{datum zkoušky:}         & 28. 4. 2014 \\
\textit{hodnocení:}             & 1 \\
\textit{cíl předmětu (STAG):}   & \\
\multicolumn{2}{p{16cm}}{Cílem předmětu je obeznámit studenty s možnostmi odhadu parametrů, náhodných veličin a náhodných procesů v podmínkách neurčitosti z apriorních informací a měřených dat.}
\end{tabular}
\end{table}

\subsection{Problémy odhadu, základní etapy vývoje teorie odhadu, náhodné veličiny, náhodné procesy a jejich popis, stochastický systém.}

\subsection{Optimální odhad ve smyslu střední kvadratické chyby. Odhad ve smyslu maximální věrohodnosti.}

\subsection{Jednorázové a rekurzivní odhady.}

\subsection{Odhad stavu lineárního diskrétního systému – filtrace (Kalmanův filtr).}

\subsection{Úlohy odhadu stavu lineárního diskrétního stochastického systému – predikce a vyhlazování.}

\subsection{Odhad stavu lineárního systému se spojitým či diskrétním měřením (Kalman-Bucyho filtr).}

\section{Optimální systémy [OPS]}

\begin{table}[H]
\centering
\begin{tabular}{p{4cm} p{12cm}}
\textit{vyučující:}             & Ing. Miroslav Flídr, Ph.D. \\
								 & Ing. Ivo Punčochář, Ph.D. \\
\textit{ročník/semestr studia:} & 4.ročník/LS \\
\textit{datum zkoušky:}         & 15. 7. 2015 \\
\textit{hodnocení:}             & 3 \\
\textit{cíl předmětu (STAG):}   & \\
\multicolumn{2}{p{16cm}}{Cílem předmětu je seznámení studentů s různými typy optimalizačních úloh. Studenti se naučí řešit jednak základní statické optimalizační úlohy tak především úlohy optimalizace dynamických systémů. Důraz je kladen především na pochopení řešení následujících problémů: 
\begin{itemize}
\item časově optimální řízení;
\item Pontrjaginův princip minima;
\item dynamické programování a Bellmanova optimalizační rekurze;
\item lineárně - kvadratická úloha optimálního řízení.
\end{itemize}}
\end{tabular}
\end{table}

\subsection{Optimální programové řízení diskrétních dynamických systémů. Formulace úlohy. Hamiltonova funkce. Nutné podmínky pro optimální řízení.}

\subsection{Optimální programové řízení spojitých dynamických systémů. Formulace úlohy. Hamiltonova funkce. Nutné podmínky pro optimální řízení. Podmínky transverzality. Pontrjaginův princip minima.}

\subsection{Deterministický diskrétní systém automatického řízení. Princip optimality. Bellmanova funkce. Bellmanova optimalizační rekurze.}

\subsection{Syntéza optimálního deterministického systému automatického řízení pro diskrétní lineární řízený systém a kvadratické kritérium. Formulace a řešení. Asymptotické řešení a jeho stabilita.}

\subsection{Deterministický spojitý systém automatického řízení. Kontinualizace Bellmanovy optimalizační rekurze.}

\subsection{Optimální stochastický systém automatického řízení. Strategie řízení. Bellmanova funkce a Bellmanova optimalizační rekurze.}

\subsection{Syntéza optimálního systému automatického řízení pro lineární gaussovský řízený systém a kvadratické kritérium. Formulace a řešení. Separační teorém.}

\section{Adaptivní systémy [AS]}

\begin{table}[H]
\centering
\begin{tabular}{p{4cm} p{12cm}}
\textit{vyučující:}             & Ing. Jindřich Duník, Ph.D. \\
								 & Ing. Ladislav Král, Ph.D. \\
\textit{ročník/semestr studia:} & 5.ročník/ZS \\
\textit{datum zkoušky:}         & 12. 12. 2016 \\
\textit{hodnocení:}             & 1 \\
\textit{cíl předmětu (STAG):}   & \\
\multicolumn{2}{p{16cm}}{Cílem předmětu je obeznámit studenty s adaptivními systémy automatického řízení a adaptivními systémy zpracování signálů.}
\end{tabular}
\end{table}

\subsection{Základní přístupy k syntéze adaptivních řídicích systémů, schematické vyjádření, srovnání s předpoklady a návrhem standardních regulátorů.}

\subsection{Adaptivní řízení s referenčním modelem, MIT pravidlo, využití Ljapunovovy teorie stability.}

\subsection{Samonastavující se regulátory, charakteristika a základní přístupy k návrhu bloku řízení, přiřazení pólů, diofantické rovnice, minimální variance.}

\subsection{Samonastavující se regulátory, charakteristika a základní přístupy k návrhu bloku poznávání, parametrické metody odhadu.}

\subsection{Adaptivní systémy na zpracování signálu, adaptivní prediktor, adaptivní filtr, analogie se samonastavujícími se regulátory.}

\subsection{Adaptivní řízení a strukturální vlastnost stochastického optimálního řízení, duální řízení, neutralita, separabilita, ekvivalence určitosti.}